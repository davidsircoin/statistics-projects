\documentclass{article}
%% Open ~/.vim/UltiSnips/tex.snippets with "gf" to change snippets (remember to ",u")
\usepackage[a4paper, total={6in, 8in}]{geometry} % Page margins
\usepackage[utf8]{inputenc}
\usepackage{amsmath, amssymb, mathtools, amsfonts, amsthm}
\usepackage{wasysym} % Smiley QED
\usepackage{eulervm} % Font
\usepackage{fancyhdr} % Custom headers and footers
\fancyhead[C]{\thepage} % Page numbering for center header
\usepackage{mdframed}
\usepackage{xcolor}

% list environments               
\usepackage{enumerate}            
\usepackage[shortlabels]{enumitem}      

% figure support
\usepackage{import}
\usepackage{xifthen}
\pdfminorversion=7
\usepackage{pdfpages}
\usepackage{transparent}
\newcommand{\incfig}[1]{%
    \def\svgwidth{\columnwidth}
    \import{./figures/}{#1.pdf_tex}
}

\pdfsuppresswarningpagegroup=1

%Commands and titles
\renewcommand\qedsymbol{\smiley}
\theoremstyle{definition}
\newtheorem*{remark}{Remark}
\theoremstyle{definition}
\newmdtheoremenv[backgroundcolor=black!4,
innertopmargin=-2pt]{theorem}{Theorem}                                           
\theoremstyle{definition}
\newmdtheoremenv[backgroundcolor=black!4,
innertopmargin=-2pt]{corollary}{Corollary}[theorem]
\newmdtheoremenv[backgroundcolor=black!4,
innertopmargin=-2pt]{lemma}[theorem]{Lemma}
\theoremstyle{definition}
\newmdtheoremenv[backgroundcolor=black!4,
innertopmargin=-2pt]{definition}{Definition}
\newmdtheoremenv[backgroundcolor=black!4,
innertopmargin=-2pt]{exercise}{Exercise}
\title{Anteckningar (och repetition av) sannolikhet och statistik}
\author{David Sermoneta}


\begin{document}
\maketitle

\section*{Sampling and estimation}

\subsection*{A few words on samples}

\subsection*{Why estimate, and how?}

\subsection*{Confidence intervals}
We have an estimate of i.e. the expected value of the sampled population's
associated distribution. Sometimes that's not good enough, and so we would
rather want to give an interval in which the true EV lies. Sampling error is 
inevitable.

\begin{definition} 
Given a sample, $x_{n} = x_1,\ldots,x_{n}$ define the
confidence interval of paramenter $\theta$, and confidence $c \in  (0,1)$ as \[
 I_{\theta, c} = \left[ \under{\theta}(x_{n}) , \overline{\theta}(x_{n}) \right],
 \]  where $P(\under{\theta}(x_{n}) \leq  \theta \leq  \overline{\theta}(x_{n}) ) \leq  c$
\end{definition}
Now for a procedure to estimate such an interval.
\begin{remark}
 \;
\begin{enumerate}[(1)]
\item Find an estimate of the sought after parameter $\theta.$ We call the estimate
$\hat{\theta}$, and it's obviously a function of the sample.
\item Since we know the distributions of our sample, treat  $\hat{\theta}$ as a
function of the random variables $X_1,\ldots,X_{n}$ each corresponding to
$x_1,\ldots,x_{n}$. Call this random variable $\Theta(X_1, \ldots,X_{n})$ and
 transform $\Theta$ (often via standardising it) into a new random variable with nice properties, that does not depend anymore on the unknown parameter  $\theta$. Call this RV $R_{\Theta}$. 
\item Now for the actual calculation. Find the quantiles $r_1, r_2$ that give you the sought after confidence number $c$, these are decided by the equation \[
P(r_1 \leq  R_{\Theta} \leq  r_2) \leq  c.
\]  Now we can solve for $\theta$, (usually as part of a standardizing transformation of an RV, we either add or multiply a factor of  $\theta$). This will look like \[
P(\under{\theta}(x_{n}) \leq  \theta \leq  \overline{\theta}(x_{n})) \leq  c.
\] And our confidence interval has been computed.
\end{enumerate}
\end{remark}


\end{proof}
\subsection*{Likelihood function}
\begin{definition}[Likelihood Function]
The \textit{likelihood function} is defined by \[
\mathcal{L}_{n} (\theta) = \prod_{i=1}^{n} f(X_{i} ; \theta) .
\] 
Similarly, the log-likelihood function is defined by $\ell_{n} (\theta) = \log
\mathcal{L}_{n}(\theta)$. 
\end{definition}

Sometimes (why?), it is of interest to maximize this function, leading to the following definition: 
\begin{definition}
 The \textit{maximum likelihood estimator} (MLE), denoted by $\hat{\theta}_{n},$ is the value of $\theta$ that maximizes $\mathcal{L}_{n}(\theta)$.
\end{definition}

\end{document}
